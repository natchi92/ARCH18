% easychair.tex,v 3.4 2016/10/19
\documentclass[EPiC]{easychair}

\usepackage{doc}
\usepackage{tikz}
\usepackage{subfigure}
\usepackage{tabularx}
\usepackage{booktabs}
\usepackage{framed}
\usepackage{amsmath,amssymb,amsfonts}
\usepackage{hyperref}
\usepackage[defaultlines=4,all]{nowidow}
\usepackage{enumitem} % GF 2017-03-28
\usepackage{tablefootnote}


\newlength{\plotwidth}
\setlength{\plotwidth}{0.45\columnwidth}
\newlength{\plotwidthShort}
\setlength{\plotwidthShort}{0.4\columnwidth}


%% - Utils
\newcommand{\todo}[1]{
  \begin{framed}
    \noindent{\bf TODO: }
    #1
  \end{framed}
}
\newcommand{\remark}[1]{
  \begin{framed}
    \noindent{\bf Remark: }
    #1
  \end{framed}
}
%\makeindex

%% Front Matter
%%
% Regular title as in the article class.
%
\title{ARCH-COMP18 Category Report:\\ Stochastic Modelling}

% Authors are joined by \and. Their affiliations are given by \inst, which indexes
% into the list defined using \institute
% Author order: first author is group leader, other authors are added alphabetically
\author{ a
       \inst{1}
%\and
 %  test \inst{2}
}

% Institutes for affiliations are also joined by \and,
\institute{ a 
  \email{aa@email.com}}

%  \authorrunning{} has to be set for the shorter version of the authors' names;
% otherwise a warning will be rendered in the running heads. When processed by
% EasyChair, this command is mandatory: a document without \authorrunning
% will be rejected by EasyChair

\authorrunning{}

% \titlerunning{} has to be set to either the main title or its shorter
% version for the running heads. When processed by
% EasyChair, this command is mandatory: a document without \titlerunning
% will be rejected by EasyChair
\titlerunning{ARCH-COMP18 Stochastic models}

\begin{document}

\maketitle

\begin{abstract}
This report presents the results of a friendly competition for formal verification of continuous and hybrid systems with linear continuous dynamics. The friendly competition took place as part of the workshop \underline{A}pplied Ve\underline{r}ification for \underline{C}ontinuous and \underline{H}ybrid Systems (ARCH) in 2018. 
\end{abstract}

%------------------------------------------------------------------------------
\section{Introduction}
\label{sect:introduction}

\begin{framed}
\paragraph{Disclaimer} The presented report of the ARCH friendly competition for \textit{stochastic modelling group} which aims at providing a landscape ....
\end{framed}

This report summarizes results obtained in the 2018 friendly competition of the ARCH workshop\footnote{Workshop on \underline{A}pplied Ve\underline{r}ification for \underline{C}ontinuous and \underline{H}ybrid Systems (ARCH), \href{http://cps-vo.org/group/ARCH}{cps-vo.org/group/ARCH}} for 
 

The goal of the friendly competition is not to rank the results, but rather to present the landscape of existing solutions in a breadth that is not possible with scientific publications in classical venues. Such publications would typically require the presentation of novel techniques, while this report showcases the current state-of-the-art tools. For all results reported by each participant, we have run an independent repeatability evaluation. 


%------------------------------------------------------------------------------
\section{Benchmarks}
\label{sec:benchmarks}

\paragraph{Types of Inputs} .

\paragraph{Different Paths to Success} When tools use a fundamentally different way of solving a benchmark problem, we add further explanations. 




%------------------------------------------------------------------------------
\section{Conclusion and Outlook}
\label{sect:conclusion}

This report presents the results on a first friendly competition for the formal verification of c


%------------------------------------------------------------------------------
\section{Acknowledgments}
\label{sec:acks}

\bibliographystyle{plain}
\bibliography{}

\end{document}



