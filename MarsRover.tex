\subsection{Mars Rover Benchmark}

We introduce the problem of planetary exploration with a two-agent
team composed of a Mars rover and a Mars helicopter, tasked to collect a set of
samples from the Mars surface.
Due to the high cost and risk associated with these missions, it is critical
to have strong correctness and performance guarantees on the robots' behaviors.
To aid this objective, we give a mission specification framework and
formally encode complex mission specifications used for real Mars rover operation.
The logic of choice is Distribution Temporal Logic (DTL) \cite{JonesDTL2013} as it captures temporal and uncertainty constraints
ranging from science campaign (e.g., getting science results from certain regions)
to operational constraints such as maintaining certain accuracy over the vehicles' state
and the environment, inter-agent distance, energy level, etc.


For a first and more accessible case study, we consider a simplified version of the full specification set.
Additionally, the rover is assumed to have stochastic motion and perception, and is tasked
with collecting samples from a partially-known deterministic environment.
The helicopter provides support in terms of exploring the environment on-demand.
Exploration is necessary if insufficient knowledge is available to synthesize a control policy with sufficient success probability.

	\subsubsection{General problem}
	\label{sec:POMDP}
	In this section, we define a language for planning under uncertainty in partially-observable environments using formal methods. We will start by reviewing partially-observable systems and the POMDP problem. Then we will go over the belief-based temporal logic for mission planning.
	
    %%%%%%%%%%%%%%%%%%%%%%%%%%%%%%%%%%%%%%%%%%%%%%%%%%%%%%%%%%
	\paragraph{POMDP.}
	Let us denote the system state, action, and observation at the $k$-th time-step by $x_k, u_k, z_k$, respectively. Let
	\begin{align}
	x_{k+1} &= f(x_k,u_k,w_k) \label{eq:dynamics} \\
	z_k &= h(x_k,v_k) \label{eq:observations}
	\end{align}
	denote the system dynamics and measurement models, where $w_k\sim p_w(\cdot|x_k)$ and $v_k\sim p_v(\cdot|x_k)$ denote the state-dependent process and observation noise. Due to the observation noise, the best one can infer about the system state is a probability distribution over all possible states, referred to as \emph{belief} $b_k=p(x_k|z_{0:k})$. The belief space (i.e., the set of all beliefs) is denoted by $\mathbb{B}$. Belief is typically evolved using a recursive filter denoted by $\tau$ as
	\begin{equation}
	\label{eq:filter}
	b_{k+1}=\tau(b_k,u_k,z_{k+1}).
	\end{equation}
   As is common, one can generally assume that the belief space is a Polish space\footnote{With Euclidean spaces as a typical example, a Polish space is a separable and completely metrizable topological space.} and that the recursive filter is a Borel measurable mapping.



   The control synthesis problem for the POMDP with DTL specifications can be formulated as follows:

\paragraph{Mars-rover problem}
    Let $\phi$ be a given DTL formula and let the robot
    evolve according to dynamics~\eqref{eq:dynamics},
    with observation dynamics~\eqref{eq:observations},
    and using a Kalman filter defined by~\eqref{eq:filter}.
    Find a policy $\mu^*$ such that 
    
    \begin{equation}
    \label{eq:mpp}
    \begin{aligned}
    &\mu^* = \underset{\mu \in \mathbb{M}}{\arg \max}Pr[\mathbf{b} \models \phi] \\
    &\text{subject to~\eqref{eq:dynamics},~\eqref{eq:observations},~\eqref{eq:filter}.}
    \end{aligned}
    \end{equation}
    
    
    The set of requirements below specify a science campaign.
	\begin{enumerate}
      \item \textit{Rover must visit any two of the science targets of type A and one of the science targets of type B}. These areas are shown in purple and blue, respectively, in Fig. \ref{fig:Scenario}. They represent areas for potentially high-science-value rock/soil samples that need to be collected.
      \item \textit{Rover must maintain the variance of its state below $\Sigma_{max}$}. This specification is to maintain accuracy along the mission and reduce the risk of unexpected events.
      \item \textit{Rover routes must avoid hazards}. Hazard areas include large rocks, deep sands, steep slopes, and are given as labeled regions on the map as part of the science campaign.
      \item \textit{Rover must follow speed constraints}.
    %   Fig.~\ref{fig:SpeedConstriants} encodes the speed constraints.
      Rover's maximum speed is a function of cumulative fractional area of rocks and the terrain slope, see Fig.~\ref{fig:SpeedConstriants}.
      \item \textit{Helicopter and Rover must satisfy battery and power constraints}. In particular, Mars helicopter can only fly for 2-3 minutes per day. Thus, the remaining battery level plays an important role in shaping the mission.
      \item \textit{Mars helicopter must avoid landing near hazards}. These hazards include large rocks, deep sands, steep slopes, etc.
      \item \textit{Helicopter must maintain a minimum distance $d_{min}$ and a maximum distance $d_{max}$ from the rover}. The purpose of this specification is to protect the main asset of the mission: the rover, which carries all important science instruments. The helicopter should not pose any risk for the rover in case of helicopter failure but should also maintain communication range.
    \end{enumerate}


 
 \subsubsection{Simple problem}
 
For simplicity, we restrict our attention to the rover and assume that the map is deterministic and the helicopter can fly to the desired location and generate a labelled map of the area requested by the rover. 
However, there can be a discrepancy between the labels in the map and the environment,
which the rover will encounter only during execution.
The campaign is defined using the following DTL specification:

\begin{equation}
    \bigwedge_{X \in \{A,B \}}  \left( \xi_{keepout} {\ \mathcal{U}\ } \xi_X \right).
\end{equation}

%\begin{equation}
%    \bigwedge_{X \in \{A,B \}}   \xi_{keepout} \Until \xi_X .
% \end{equation}
The dynamics of the rover are represented by a unicycle model
\begin{equation}
    \begin{bmatrix}
        x(t+1) \\
        y(t+1) \\
        \varphi(t+1)
    \end{bmatrix} = \begin{bmatrix}x(t) + v \cos(\varphi(t)) \delta{t}\\ y(t) + v \sin(\varphi(t)) \delta{t}\\ \varphi(t) + \omega \delta{t} \end{bmatrix} + w,
\end{equation}
where $v$ (velocity) and $\omega$ (angular velocity) are the inputs and $w \sim \mathcal N(0, Q)$ is Gaussian process noise. 
\todo{Mars rover benchmark: partially added, DTL should be added. More details on case study are needed.}